\hypertarget{libphonefirewall_8h}{
\section{libphonefirewall.h File Reference}
\label{libphonefirewall_8h}\index{libphonefirewall.h@{libphonefirewall.h}}
}
API of the phone firewall. 



This graph shows which files directly or indirectly include this file:\nopagebreak
\begin{figure}[H]
\begin{center}
\leavevmode
\includegraphics[width=104pt]{libphonefirewall_8h__dep__incl}
\end{center}
\end{figure}
\subsection*{Defines}
\begin{CompactItemize}
\item 
\#define \hyperlink{libphonefirewall_8h_f0f2173e3b202ddf5756531b4471dcb2}{MAX\_\-LINE\_\-LENGTH}~512
\end{CompactItemize}
\subsection*{Functions}
\begin{CompactItemize}
\item 
int \hyperlink{libphonefirewall_8h_ca47a7a15cc44ee9c44ccbaf61ab63b7}{add\_\-blacklist\_\-entry} (int country\_\-code, int area\_\-code, long long int number, char $\ast$name, char $\ast$reason, int priority)
\item 
int \hyperlink{libphonefirewall_8h_c1875490cf592d507798a292f8a772e9}{rm\_\-blacklist\_\-entry} (long long int number)
\item 
int \hyperlink{libphonefirewall_8h_979a7b693ec5468f9caa2487bd1b7217}{add\_\-whitelist\_\-entry} (int country\_\-code, int area\_\-code, long long int number, char $\ast$name, char $\ast$reason, int priority)
\item 
int \hyperlink{libphonefirewall_8h_c3a77117dbcade02ab233836e559814a}{rm\_\-whitelist\_\-entry} (long long int number)
\item 
char $\ast$ \hyperlink{libphonefirewall_8h_737cba40ba261f05830a454be982f56e}{check\_\-blacklist\_\-entry} (int country\_\-code, int area\_\-code, long long int number)
\item 
char $\ast$ \hyperlink{libphonefirewall_8h_db8b63f034438c47d82a80ed205f0ad2}{check\_\-whitelist\_\-entry} (int country\_\-code, int area\_\-code, long long int number)
\end{CompactItemize}


\subsection{Detailed Description}
API of the phone firewall. 

\begin{Desc}
\item[Author:]Alex Oberhauser\end{Desc}
The header file of the Phone Firewall. Blocks or accepts incoming phone calls, so it's possible to prevent disturbing phone calls. Provides a API which can used by other application to build nice programs.

Implemented for the OpenMoko framework. 

Definition in file \hyperlink{libphonefirewall_8h-source}{libphonefirewall.h}.

\subsection{Define Documentation}
\hypertarget{libphonefirewall_8h_f0f2173e3b202ddf5756531b4471dcb2}{
\index{libphonefirewall.h@{libphonefirewall.h}!MAX\_\-LINE\_\-LENGTH@{MAX\_\-LINE\_\-LENGTH}}
\index{MAX\_\-LINE\_\-LENGTH@{MAX\_\-LINE\_\-LENGTH}!libphonefirewall.h@{libphonefirewall.h}}
\subsubsection{\setlength{\rightskip}{0pt plus 5cm}\#define MAX\_\-LINE\_\-LENGTH~512}}
\label{libphonefirewall_8h_f0f2173e3b202ddf5756531b4471dcb2}




Definition at line 33 of file libphonefirewall.h.

Referenced by check\_\-blacklist\_\-entry().

\subsection{Function Documentation}
\hypertarget{libphonefirewall_8h_ca47a7a15cc44ee9c44ccbaf61ab63b7}{
\index{libphonefirewall.h@{libphonefirewall.h}!add\_\-blacklist\_\-entry@{add\_\-blacklist\_\-entry}}
\index{add\_\-blacklist\_\-entry@{add\_\-blacklist\_\-entry}!libphonefirewall.h@{libphonefirewall.h}}
\subsubsection{\setlength{\rightskip}{0pt plus 5cm}int add\_\-blacklist\_\-entry (int {\em country\_\-code}, int {\em area\_\-code}, long long int {\em number}, char $\ast$ {\em name}, char $\ast$ {\em reason}, int {\em priority})}}
\label{libphonefirewall_8h_ca47a7a15cc44ee9c44ccbaf61ab63b7}


Add a number to the blacklist. The number will be blocked after that.

\begin{Desc}
\item[Parameters:]
\begin{description}
\item[{\em country\_\-code}]The country code (for example 39 for Italy, 43 for Austria, and so one) \item[{\em area\_\-code}]The area code which indicates your mobile operator. \item[{\em number}]The telephone number of the person. \item[{\em name}]The name of the person. \item[{\em reason}]Why you have blocked this person. \item[{\em priority}]Has no affect at the moment. Later one it will be possible to give each number priority. So you have more control when a number will be blocked/accepted.\end{description}
\end{Desc}
\begin{Desc}
\item[Returns:]If all goes well 0 (zero) otherwise an errno code. \end{Desc}


Definition at line 31 of file phonefirewall\_\-administration.c.

References DELIM.\hypertarget{libphonefirewall_8h_979a7b693ec5468f9caa2487bd1b7217}{
\index{libphonefirewall.h@{libphonefirewall.h}!add\_\-whitelist\_\-entry@{add\_\-whitelist\_\-entry}}
\index{add\_\-whitelist\_\-entry@{add\_\-whitelist\_\-entry}!libphonefirewall.h@{libphonefirewall.h}}
\subsubsection{\setlength{\rightskip}{0pt plus 5cm}int add\_\-whitelist\_\-entry (int {\em country\_\-code}, int {\em area\_\-code}, long long int {\em number}, char $\ast$ {\em name}, char $\ast$ {\em reason}, int {\em priority})}}
\label{libphonefirewall_8h_979a7b693ec5468f9caa2487bd1b7217}


Add a number to the whitelist. The number will be accepted after that.

\begin{Desc}
\item[Parameters:]
\begin{description}
\item[{\em country\_\-code}]The country code (for example 39 for Italy, 43 for Austria, and so one) \item[{\em area\_\-code}]The area code which indicates your mobile operator. \item[{\em number}]The telephone number of the person. \item[{\em name}]The name of the person. \item[{\em reason}]Why you have blocked this person. \item[{\em priority}]Has no affect at the moment. Later one it will be possible to give each number priority. So you have more control when a number will be blocked/accepted.\end{description}
\end{Desc}
\begin{Desc}
\item[Returns:]If all goes well 0 (zero) otherwise an errno code. \end{Desc}


Definition at line 47 of file phonefirewall\_\-administration.c.

References DELIM.\hypertarget{libphonefirewall_8h_737cba40ba261f05830a454be982f56e}{
\index{libphonefirewall.h@{libphonefirewall.h}!check\_\-blacklist\_\-entry@{check\_\-blacklist\_\-entry}}
\index{check\_\-blacklist\_\-entry@{check\_\-blacklist\_\-entry}!libphonefirewall.h@{libphonefirewall.h}}
\subsubsection{\setlength{\rightskip}{0pt plus 5cm}char$\ast$ check\_\-blacklist\_\-entry (int {\em country\_\-code}, int {\em area\_\-code}, long long int {\em number})}}
\label{libphonefirewall_8h_737cba40ba261f05830a454be982f56e}


Checks if a number is on the blacklist.

\begin{Desc}
\item[Parameters:]
\begin{description}
\item[{\em country\_\-code}]The country code (for example 39 for Italy, 43 for Austria, and so one) \item[{\em area\_\-code}]The area code which indicates your mobile operator. \item[{\em number}]The telephone number of the person.\end{description}
\end{Desc}
\begin{Desc}
\item[Returns:]If noting is found NULL, otherwise the number. \end{Desc}


Definition at line 71 of file phonefirewall\_\-administration.c.

References DELIM, and MAX\_\-LINE\_\-LENGTH.\hypertarget{libphonefirewall_8h_db8b63f034438c47d82a80ed205f0ad2}{
\index{libphonefirewall.h@{libphonefirewall.h}!check\_\-whitelist\_\-entry@{check\_\-whitelist\_\-entry}}
\index{check\_\-whitelist\_\-entry@{check\_\-whitelist\_\-entry}!libphonefirewall.h@{libphonefirewall.h}}
\subsubsection{\setlength{\rightskip}{0pt plus 5cm}char$\ast$ check\_\-whitelist\_\-entry (int {\em country\_\-code}, int {\em area\_\-code}, long long int {\em number})}}
\label{libphonefirewall_8h_db8b63f034438c47d82a80ed205f0ad2}


Checks if a number is on the whitelist.

\begin{Desc}
\item[Parameters:]
\begin{description}
\item[{\em country\_\-code}]The country code (for example 39 for Italy, 43 for Austria, and so one) \item[{\em area\_\-code}]The area code which indicates your mobile operator. \item[{\em number}]The telephone number of the person.\end{description}
\end{Desc}
\begin{Desc}
\item[Returns:]If noting is found NULL, otherwise the number. \end{Desc}


Definition at line 100 of file phonefirewall\_\-administration.c.\hypertarget{libphonefirewall_8h_c1875490cf592d507798a292f8a772e9}{
\index{libphonefirewall.h@{libphonefirewall.h}!rm\_\-blacklist\_\-entry@{rm\_\-blacklist\_\-entry}}
\index{rm\_\-blacklist\_\-entry@{rm\_\-blacklist\_\-entry}!libphonefirewall.h@{libphonefirewall.h}}
\subsubsection{\setlength{\rightskip}{0pt plus 5cm}int rm\_\-blacklist\_\-entry (long long int {\em number})}}
\label{libphonefirewall_8h_c1875490cf592d507798a292f8a772e9}


Removes a blocked number from the blacklist. 

Definition at line 63 of file phonefirewall\_\-administration.c.\hypertarget{libphonefirewall_8h_c3a77117dbcade02ab233836e559814a}{
\index{libphonefirewall.h@{libphonefirewall.h}!rm\_\-whitelist\_\-entry@{rm\_\-whitelist\_\-entry}}
\index{rm\_\-whitelist\_\-entry@{rm\_\-whitelist\_\-entry}!libphonefirewall.h@{libphonefirewall.h}}
\subsubsection{\setlength{\rightskip}{0pt plus 5cm}int rm\_\-whitelist\_\-entry (long long int {\em number})}}
\label{libphonefirewall_8h_c3a77117dbcade02ab233836e559814a}


Removes a accepted number from the whitelist. 

Definition at line 67 of file phonefirewall\_\-administration.c.